\documentclass{article}

\author{Yining Wang}
\title { \LARGE \bf Understanding of cryo-EM: \\ \bf  Image formation and 3D reconstruction\\ \normalsize written by \LaTeX{} }

\usepackage{graphicx}
\usepackage[colorlinks, linkcolor = black]{hyperref}
\usepackage{ulem}
\usepackage{amsmath}
\usepackage{amssymb}
\usepackage{framed}
\usepackage{bookmark}
\usepackage{geometry}

\geometry{a4paper,scale=0.75}




\begin{document}

    \maketitle
    \thispagestyle{empty}
    \begin{figure}[h]
        \centering   
        \includegraphics[scale = 0.17]{/Users/yiningwang/Desktop/HMWGS.PNG} 
    \end{figure}

   
    \newpage
    \tableofcontents


    \newpage
    \thispagestyle{empty}
    \begin{figure}[p]
        \centering
        \includegraphics[scale = 0.09]{/Users/yiningwang/Desktop/HMWGS.PNG} 
        \caption{\small HMW-GS predicted by AlphaFold(Pipeline)}
    \end{figure}
    \begin{figure}[p]
        \centering
        \includegraphics[scale = 0.09]{/Users/yiningwang/Desktop/AlphaFold/pics/LMW.png} 
        \caption{\small LMW-GS predicted by AlphaFold(Colab version)}
    \end{figure}
    \begin{figure}[p]
        \centering
        \includegraphics[scale = 0.09]{/Users/yiningwang/Desktop/AlphaFold/pics/gli.png} 
        \caption{\small Gliadin predicted by AlphaFold(Colab version)}
    \end{figure}
    
    \newpage




    \begin{abstract}
        \bf This one is a brief summary of my two-month internship at IGDB of Chinese Acadamy of Science
        It is a great experience and I really apperatiate the opportunity to learn some the cutting-edge
        technology like cryo eletron microscope and other bioinfomatic techniques \\
        Other techniques when applying cryo-EM are relatively easier to understand so in this note I laid more importance on 
        those parts which are more obscure.\\ In other words, this note states mostly about the theoratical part about 
        image formation and the basic operation of cryo-EM\\After all, the whole process of this study will last
        for a long period of time, hopefully we can get a good result.

    \end{abstract}
    %\thispagestyle{}
    \newpage

    \section{cryo-EM versus Negative-staining EM}

    $\blacktriangleright $ Using Negative-staining EM to get a initial model and then using cryo-EM to 
    REFINE the model\\
    $\blacktriangleright $ Not all structures need initial model!(symmetric or high-quality enough or in a nice shape!)\\
    $\blacktriangleright $ The sample preparation process is much easier using ns-EM
    \begin{figure}[ht]
        \centering
        \includegraphics[scale = 1]{/Users/yiningwang/Desktop/ns.png} 
        \caption{\small Negative-staining EM}
    \end{figure}

    
    \section{Evaluation Criteria during structure anlysis using cryo-EM}
    \subsection{Protein preparation}
    \begin{itemize}
        \item Composition
        \item \bf Purity and homogenity
        \item \bf Stability (buffer composition)
        \item Biochemical activity
    \end{itemize}
    
    \subsection{Negative staining}
    \begin{itemize}
        \item Discrate particle
        \item Stability
        \item  \bf Particle size and particle shape
    \end{itemize}
    \subsection{Diagnostic cryo-EM}
    \begin{itemize}
        \item Stability
        \item Particle size and shape
        \item \bf concentration and particle distribution
    \end{itemize}
    \subsection{Initial cryo-EM data collection}
    \begin{itemize}
        \item \bf High-resolution 2D classes
        \item Initial 3D model 
        \item Orientation and distribution
        \item Particle yield
    \end{itemize}
    \subsection{High-resolution cryo-EM data collection}
    \begin{itemize}
        \item Tilt pairs
        \item Motion statistics
        \item Angular accuracy
        \item Conformational states
    \end{itemize}
















    \section{Image processing and 3-D reconstruction}

    \subsection{Fourier Transform}

    \subsubsection{Fourier Transform}
    \begin{figure}[ht]
        \centering   
        \includegraphics[scale = 0.3]{/Users/yiningwang/Desktop/3.png} 
        \caption{\small Fourier transform}
    \end{figure}
    Fourier Transform allows us to describe a specific real-space matter in the frequency domain 
    It answer the question that how do the density values of each point in space overlay. Hence it give
     us a new angle to understand familiar substance.
    Composition of different spatial frequencies forms a image
    real space $\leftrightarrows $ reciprocal space
    There are two main information convied by a WAVE :Amplitude($\mathcal{A} $) and Phase shift($\varphi$), and through a fourier Transform
    these two element could be linked.\\
    \begin{center}
        \begin{framed}
            $F = \left\lvert A\right\rvert \cdot e^{i\varphi_F} $  
        \end{framed}
    \end{center}
    The process of image formation is a process of FOURIER TRANSFORM !
    \subsubsection{Nyquist Frequency}
    The highest frequency information in the image, The magnitude of the change cannot exceed the size of one pixel.
    If the size of a pixel is 2Å, as a result the smallest wavelength can be described is 4Å. For example, if we want to get a resolution of 1 $angstrom$
     the pixel size needed is 0.5 $angstrom$
     
    \subsubsection{Convolution}
    Every spot on the image becomes a Convolution kernel(CTF in this case) and then stacked together.
    \begin{figure}[h]
        \centering
        \includegraphics[scale = 0.4]{/Users/yiningwang/Desktop/con1.png} 
        \caption{\small Convolution theorem 01}
    \end{figure}
    \begin{figure}[ht]
        \centering
        \includegraphics[scale = 0.3]{/Users/yiningwang/Desktop/con2.png} 
        \caption{\small Convolution theorem 02}
    \end{figure}
    \begin{center}
        \begin{framed}
            $h(x) = f(x) \otimes   g(x) = \int_{\infty}^{-\infty} f(t)g(x - t) \,dx $ \\ 
            $V(x) \otimes P(x) = Im(x)$
        \end{framed}
    \end{center}
    $P(x)$:Point scatter function -- This related only to the imaging system, it has nothing to do with the sample.\\
    The Fourier transform of the convolution of two 
    functions is the product of their respective Fourier transforms
    \subsection{Image filtering}
    Take FFT $\Longrightarrow$ Multiplied by filter $\Longrightarrow$ IFT
    \begin{itemize}
        \item High-pass filter: Good for high resolution details
        \item Low-pass filter: Increase contrast and SNR
        \item Bond-pass filter: Contrast Transform Funtion 
    \end{itemize}
    \subsection{Centeral Section theorem and 3-D reconstruction}
    There is an exact relationship between Fourier transform and its mathematical projection\\
    \begin{figure}[h]
        \centering
        \includegraphics[scale = 0.8]{/Users/yiningwang/Desktop/2.jpg} 
        \caption{\small Centeral Section theorem}
    \end{figure}
    \subsection{Common model for information restoration }
    \begin{center}  
         $X = PSF \otimes P_\varphi V \otimes MTF + Noise$
    \end{center}
    As it is shown above, the main problem for us is to solve the parameter V form X. 
    Under our specific circumstance the function is below
    \begin{center}
        \begin{framed}
            \begin{itemize}
                \item real space:  $X = PSF \otimes P_\varphi V \otimes MTF + N$
                \item reciprocal space: $X = CTF \cdot \hat{P_\varphi}\hat{V} \cdot MTF +\hat{N} $
                \item $CTF = sin[\Pi \cdot \lambda \cdot \varDelta f \cdot s^2 + \frac{\pi \cdot 
                \lambda^3 \cdot C_s \cdot s^4}{2}  ]$
            \end{itemize}  
           
        \end{framed}
    \end{center}
    $\hat{P_\varphi}$ can be solved only when the ORENTATION can be determined!
    \subsection{Orientation Determination}
    Maximum likelihood \& Bayesian method
    The common line technique and Projetion matching

    \section{Single Particle Analysis: Workflow}
    In this section, each part of the SPA will be noted carefully. Main subsections range from 
    evaluation of the micrograph to the final 3D refinement of the model!
    \subsection{Micrographs evaluations}
    There are sevaral common phenomenon in this part. To make evaluation about the micrograph
    the very first thing to get is the POWER SPECTRUM of the photo, any slight drift, astigmatism or 
    loss of resolution can be discovered clearly. Also, if there are many THON rings, it means the ice layer
    is way too thin, on the oppsite, if the number of rings are too low, it means you got a thick layer of ice.
    More rings usually means that the high-resolution details of the image is great, but CAUTION , the thin layer 
    of ice may also brings damage to the specimen.
    \begin{figure}[h]
        \centering   
        \includegraphics[scale = 0.34]{/Users/yiningwang/Desktop/micro.png} 
        \caption{\small Drift, astigmatism and loss of resolution}
    \end{figure}

    \subsection{CTF estimation of each micrograph}
    \begin{center}
        \begin{framed}
            $CTF = sin[\Pi \cdot \lambda \cdot \varDelta f \cdot s^2 + \frac{\pi \cdot 
            \lambda^3 \cdot C_s \cdot s^4}{2}]$
        \end{framed}     
    \end{center}
    \subsubsection{Key factors}
    There are few key factors which determine the Contrast Transfer Function.
    \begin{center}
         \begin{itemize}
            \item Power spectrum
            \item Microscope parameter (Kv, Cs, pixel size...)
            \item Defocus($\varDelta f$) $\longrightarrow$ Delocalization
        \end{itemize}  
    \end{center}
    \subsubsection{CTF estimation}
    As the function shown above the main parameter in solving CTF is $\varDelta f$ and
    Astigmatism, to solve these two items, the method used is to compare the CTF image to
    the Thon Ring Pattern (Exhaustively), until a perfect match is complete!
    \begin{figure}[h]
        \centering   
        \includegraphics[scale = 0.3]{/Users/yiningwang/Desktop/esti.png} 
        \caption{\small CTF estimation}
    \end{figure}
    \subsubsection{CTF correction}
    \begin{center}
        \begin{framed}
            $-1 = e^{i\pi} (180 ^{\circ} phase shift) $
        \end{framed}     
    \end{center}
     CTF can be corrected by phase flipping(adding 180$^o$ to the phases) after measurement of $\varDelta f$
     But a dataset should comprise images within an adequate range of defocus in order to recover
     informations in zones of zero-crossing of CTF
    
    $\blacktriangleright $PHASE FLIPPING
    \begin{center}
        $X (s)=CTF(s)\cdot F(S) +N(S)$\\
        $M(s)=CTF(s)^2 \cdot F(s) + CTF(s) \cdot N(s)$
    \end{center}
    $\blacktriangleright $WEINER FILTER
    \begin{center}
        \begin{framed}
            $M(s)=X(s) \cdot \frac{CTF(s)}{CTF(s)^2 + {1}/{SNR} } $\\
            $SNR=\sigma^2(F(s))/\sigma^2(N(s))$
           [$SSNR(s)=\frac{M(s)^2-N(s)^2}{N(s)^2}$]

        \end{framed}
    \end{center}    
    \begin{figure}[h]
        \centering   
        \includegraphics[scale = 0.24]{/Users/yiningwang/Desktop/spa.png} 
        \caption[Significant effect of defocus]{\small Small variation in defocus have significant effect in High-resolution details}
    \end{figure}
    \subsection{Particle picking}
    \begin{center}
        \begin{itemize}
           \item Manual particle selection \& Autopick
           \item Box Size (1.5--2 time $\rightarrow $ good for FT)
           \item cross-correlation
       \end{itemize}  
   \end{center}
    \subsection{2D Image Alignment and classification}
    \subsubsection{Aims}
    \begin{center}
        \begin{itemize}
           \item clean the dataset
           \item increase SNR to generate averaged micrographs
           \item classification to purify particles
       \end{itemize}  
   \end{center}
   \subsubsection{Methods}
   \begin{center}
    \begin{enumerate}
       \item 2D Image Rotation Alignment (rotation alignment)
       \item Maximum likelihood
   \end{enumerate} 
\end{center}

    \subsection{Initial model problem}
    Initial model problem plays a fundamental part in the image formation, because the following 
    part(Projetion Mapping)
    needs a appropriate initial model to make 3D refinement, the main methods are below:
    \begin{center}
        \begin{itemize}
           \item Random Conical Tilt (Missing cone probably)
           \item Class average and Commo line method  (Good SNR needed)
           \item Electron Tomography and sub-tomo average
       \end{itemize} 
    \end{center}

    \subsection{3D refinement}
    Major methods used in 3D refinement are Projection mapping and Bayesian Method 
    \begin{figure}[h]
        \centering
        \includegraphics[scale = 0.4]{/Users/yiningwang/Desktop/PM.png} 
        \caption{\small Iterative Refinement--Projection Mapping}
    \end{figure}
    \subsection{Resolution assessment and map validation}
    Once a concentration map is built, the jod to do is to make assessment of the resolution and 
    furthermore, apply the map validation, if the resolution of the image is high enough and the side chains
    can be determined, the validation of the map can be done easily. But we can not always get hi-resolution images
    so the proper methods and algorithms to do resolution assessmet and map validation is needed
    \begin{center}
        \begin{framed}
            $F_L \cdot {F_R}^* = \left\lvert F_L\right\rvert \left\lvert F_R\right\rvert - e^{i \cdot (\varphi_L - \varphi_R)}$
        \end{framed}
    \end{center}  
    The function above is a simplified version of Fourier Shell coefficient, by using this method the resolution of the 
    initial model can be culculated. If the resolution isn't estimated correctly, a phenomenon called over fitting
    will come up and affect further section. 
    PARTICLE NUMBER DOES MATTER!!!Because the number of the particles will determine Singnal-to-Noise Rate and in advance, determine the 
    resolution, and there is a simple way to estimate the number of particle needed to reach a
    certain resolution by using the diameter of the particle 
    \begin{center}
        \begin{framed}
            Num = $\frac{\pi \cdot D}{(m) \cdot Res} \times 100\times 5$
        \end{framed}
    \end{center} 
    Once the resolution assessment is done the next part is to apply map validation, one of the mainstream 
    method is called Tilt pair Validation. 
    \begin{figure}[p]
        \centering
        \includegraphics[scale = 0.4]{/Users/yiningwang/Desktop/FSC.png} 
        \caption{\small Fourier shell coefficient}
    \end{figure}
    \begin{figure}[p]
        \centering
        \includegraphics[scale = 0.4]{/Users/yiningwang/Desktop/TPV.png} 
        \caption{\small Tilt Pair Validation}
    \end{figure}
    \subsection{3D Classification}
    Although the process stated above can be done by computer programs like RELION, it is not that easy because the 
    exsistance of two types of heterogeneity: Composition heterogeneity and Conformational heterogeneity.
    Because the SPA in valide under the hypothesis of all the particles are excatly SAME, so special computational 
    strategies are needed to solve these two problems, MDA(Multivariate Date Analysis) is a prevailing one.
    Discrete-state heterogeneity(apple/pear) is often easier to deal with than Continuous-state (green apple/light-green apple)
    \subsubsection{Discrete-state methods}
    \begin{center}
        \begin{itemize}
           \item Multireference classification
           \item Maximum likelihood methods
           \item Statistical Analysis methods
       \end{itemize} 
    \end{center}
    \subsubsection{Continuous-state methods}
    There is a clear relation between the possibility of a electron's apperance on certain spot and it's energy
    \begin{center}
        \begin{framed}
            $P\propto e^{-\frac{E}{kT}}$
        \end{framed}
    \end{center} 
    There are two methods to deal with the problem
    \begin{center}
        \begin{enumerate}
           \item Hybrid EM NMA 
           \item Manifold Embedding Method
       \end{enumerate} 
    \begin{figure}[ht]
        \centering
        \includegraphics[scale = 0.4]{/Users/yiningwang/Desktop/HEM.png} 
        \caption{\small HEMNMA}
    \end{figure}
    \end{center}
    \begin{figure}[ht]
        \centering
        \includegraphics[scale = 0.4]{/Users/yiningwang/Desktop/mem.png} 
        \caption{\small Manifold Embedding Method}
    \end{figure}
   
    



    





    \end{document}
